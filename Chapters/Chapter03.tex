%*************************************************************
\chapter{Designtheorie}\label{ch:designTheorie}
%*************************************************************

lorem ipsum

\section{Was ist Design?}
Wirft man in einen Raum voller Fachleute verschiedener Abteilungen oder Berufsgruppen die Frage nach einer Definition von Design auf, so wird jeder eine andere Antwort finden. Die Definitionen würden so breit gefächert sein, dass geradezu jede Art der Kreation – vom Schreiben eines Programms bis zum Erstellen eines Businessplans – als Design verstanden werden kann. Durch diese Vielseitigkeit, ist es schwer eine präzise, nutzvolle Beschreibung zu finden.  Hier beginnt auch die Problematik. \citep{sagmeister:2008}

\medskip 
\begin{quote}
	\begin{flushright}{\slshape    
	    Design is to design a design to produce design.} \\ \medskip
	    --- \defcitealias{heskett:2005}{John Heskett}\citetalias{heskett:2005} \citep{heskett:2005}
	\end{flushright}
\end{quote}

\subsection{Design als Prozess}
\subsection{Guiding Principles}
\subsection{Design als Repräsentation}
\subsection{Designvokabular}
\subsubsection{Storytelling}
\subsubsection{Indexical Expressions}
\subsection{Designdisziplinen}

\section{Designmethoden}
\subsection{Artefakte}
\subsubsection{Conversational Props}
\subsubsection{Boundary Objects}
\subsection{Skizzieren}
\subsection{Prototyping}
\subsection{User Testing}
\subsection{Personas \& Szenarien}

\section{Design Patterns}\label{sec:designPatterns}
\subsection{History of Patterns}
\subsection{Patterns Beispiele}

\section{Beispiel einer Designsession}

\section*{Zusammenfassung}
lorem ipsum