%*************************************************************
\chapter{CSCW \& Design}\label{ch:CSCWDesign}
%*************************************************************

	Der Begriff >>Computer Supported Cooperative Work<< (CSCW) bezeichnet ein multidisziplinäres Forschungsgebiet und existiert seit den frühen achtziger Jahren. Um herauszufinden, wie die Technik Menschen bei ihrer Zusammenarbeit unterstützen kann, organisierten Irene Greif und Paul Cashman im Jahre 1984 einen Workshop für Personen, die sich mit der Arbeitsweise von Menschen auseinandersetzten \citep{Grudin:1994}. Unter diesen Personen befanden sich Spezialisten aus verschiedenen wissenschaftlichen Bereichen, wie zum Beispiel Ökonomie, Sozialpsychologie, Anthropologie, Ethnologie und Pädagogik. Dieser Workshop war der Versuch der Techniker, Teamarbeit besser zu verstehen, um folglich unterstützende Technologien entwickeln zu können \citep{Grudin:1994, Rama:2006p245}. Seither hat CSCW sich zu einem nahezu riesigen wissenschaftlichen Forschungsgebiet entwickelt, dem sich heute unzählige Experten widmen. Trotzdem sind sich die Wissenschaftler nicht immer ganz einig bei der Definition des Begriffes CSCW. Die Bedeutung von >>Cooperative Work<< erscheint nicht eindeutig und führt häufig zu unterschiedlichen Interpretation seitens wissenschaftlicher Autoren \citep{Gerlicher:2007p241}. Einige setzen >>Cooperation<< gleich mit >>Collaboration<<, andere hingegen unterscheiden die beden Begriffe sehr strikt. Dillenbourg et al. definieren die Begriffe beispielsweise so: 
	
	\bigskip\emph{>>Cooperation and collaboration do not differ in terms of whether or not the task is distributed, but by virtue of the way in which it is divided; in cooperation the task is split (hierarchically) into independent subtasks; in collaboration cognitive processes may be (heterarchically) divided into intertwined layers. In cooperation, coordination is only required when assembling partial results, while collaboration is “...„ a coordinated, synchronous activity that is the result of a continued attempt to construct and maintain a shared conception of a problem.<<} \citep{Dillenbourg:1995}
	
	\bigskip Die Erkenntnisse, welche die Erforschung von CSCW liefert, werden dazu verwendet, sinnvolle >>Groupware<< zu konzipieren. Groupware bezeichnet also die technische Umsetzung, die auf CSCW basiert. Alle Systeme, Applikationen und Werkzeuge, die CSCW unterstützen, können somit unter dem Begriff Groupware zusammengefasst werden \citep{Koch2008, Gerlicher:2007p241}. Häufig werden diese Systeme auch als >>kollaborative Software<< bezeichnet \citep{Bannon:1990p244}. 
	
	

\begin{itemize}
	\item {CSCW: four characters in search of a context}
	\item {CSCW an Enterprise 2.0 - towards an integrated perspective}
	\item {A survey and comparison of CSCW groupware applications}
	\item {CSCW - kollaborative Systeme und Anwendungen}
\end{itemize}

\section{Klassifikation von CSCW}
\subsection{Asynchrone Systeme}
\subsection{Synchrone Systeme}
\subsection{Multisynchrone Systeme}

\section{Wichtige Aspekte bei Design von Groupware}
\begin{itemize}
	\item {Helping CSCW Applications Succeed: The Role of Mediators in the Context of Use}
	\item {How Can Human and Design Sciences Cooperate in CSCW?}
	\item {CSCW as a basis for interactive design semantics}
	\item {Developing CSCW Tools for Idea Finding - Empirical Results and Implications for Design}
	\item {Design for Design: Support for Creative Practice in CSCW in Design}
	\item {Anforderungen an interaktive Kooperationslandschaften für kreatives Arbeiten und erste Realisierungen}
	\item {Making Sense of Collaboration: The Challenge of Thinking Together in Global Design Teams}
	\item {Single Display Groupware: A Model for Co-present Collaboration}
	\item {Synergy: A Prototype Collaborative Environment to Support the Conceptual Stages of the Design Process}
	\item {Architecture of BEACH: The Software Infrastructure for Roomware Environments}
	\item {A Multiple Device Approach for Supporting Whiteboard-based Interactions}
	\item {Avoiding Interference: How People use Spatial Separation and Partitioning in SDG Workspaces}
	\item {Being Here: Designing for Distributed Hands-On Collaboration in Blended Interaction Spaces}
	\item {Beyond the Chalkboard: Computer Support for Collaboration and Problem Solving in Meetings}
\end{itemize}

\section{Warum Groupware unseren Erwartungen oft nicht gerecht wird}
\begin{itemize}
	\item {The Productivity Paradox: Why hasn't Information Technology Fulfilled its Promise?}
	\item {Why CSCW Applications Fail: Problems in the Design and Evaluation of Organizational Interfaces}
\end{itemize}

\section{Der positive Nutzen von gut umgesetzter Groupware}

\section*{Zusammenfassung}