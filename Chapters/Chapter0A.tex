%********************************************************************
% Appendix
%*******************************************************
\chapter{Gespr{\"a}che mit Designern}\label{cha:interviews}
Der folgende Abschnitt umfasst Transkripte, von uns geführten Interviews mit Designern aus verschiedenen Designsparten. Sie bieten Einblicke in Arbeitssituationen und Arbeitsweisen in den Bereichen \emph{Produktdesign}, \emph{Interaktionsdesign} und \emph{Webdesign} bzw. Erfahrungen und Anmerkungen zu Scribbler.

\medskip \emph{Anmerkung: Die Abschrift und alle dazugehörigen Materialen, wurden von den jeweiligen Interviewpartnern gestattet. Unter Rücksicht auf Anonymität werden lediglich Vornamen oder Spitznamen zur Unterscheidung verwendet. Von Thomas Nägele und Clemens Sagmeister werden stets die Vornamen verwendet.}

\section{Interview mit einem Produktdesigner}
\emph{Peter} ist von Beruf Produktdesigner und arbeitet zusätzlich als Professor an der Universität für angewandte Kunst Wien. 

\medskip \emph{Thomas}: Bitte erklär uns doch erst ein mal, wie du arbeitest, sprich, wie du ein bestimmtes Projekt abwickelst.

\medskip \emph{Peter}: Also es beginnt grundsätzlich so: du kriegst einen Auftrag vom Kunden, ein Briefing. Dort stehen die Anforderungen an dich als Designer, in welche Richtung die Arbeit gehen soll. Danach gibt es eine zwei bis drei Wochen lange Phase, in der du nur Ideen generierst. Ich mache in dieser Phase viele, relativ grobe Skizzen mit einem Kugelschreiber auf Papier, um mich an das Thema anzunähern und gewisse Ideen mal aufzuzeichnen. Die Ideenfindung passiert ja meist über die Zeichnung, das bedeutet man zeichnet irrsinnig viel, ca. 50 Blätter Papier und durch das Visualisieren ergeben sich neue Ideen. Im nächsten Stadium verfeinert man dann einige Skizzen, koloriert sie etwa mit Photoshop und gewisse Details herausarbeitet oder man geht direkt schon über auf das Arbeiten mit digitalen Mitteln. Es besteht natürlich die Gefahr, dass man zu weit ins Detail geht obwohl das in dieser Phase noch niemanden interessiert, da muss man aufpassen. 

\medskip \emph{Thomas}: Arbeitest du dann mit Pen und Tablet?

\medskip \emph{Peter}: Ja genau, am Rechner arbeite ich eigentlich nur selten mit der Maus. Hauptsächlich Pen und Tablet. Photoshop und Illustrator ausschließlich mit Pen und Tablet.

Gut, wenn man dann eine ganze Reihe an Ideen und recherchiert hat, beginnt man die Präsentation vorzubereiten. Das Ganze mache ich in InDesign, dort hab ich bereits meine Vorlagen und füge dann Texte und Bilder ein und ich versuche das so zu gestalten, dass eine gedruckte Version der Präsentation selbsterklärend ist und der Kunde die Möglichkeit hat, sich das später noch mal anzusehen. Bei der Präsentation selbst kommt ja immer so viel Information in kurzer Zeit auf die Hörer zu, dass es gar nicht möglich ist, alles vollständig aufzufassen. Daher muss man sich das nachträglich auch in Ruhe noch mal ansehen können.

\medskip \emph{Thomas}: Verteilst du die Ausdrucke dann schon zu Beginn oder wartest du bis nach der Präsentation?

\medskip \emph{Peter}: Das kommt auf die Anzahl der Teilnehmer an. Bei größeren Gruppen muss man ja einen Beamer verwenden, da warte ich bis nach der Präsentation, wenn es hingegen wenig Leute sind, dann arbeite ich sehr gerne nur mit den Ausdrucken und präsentiere direkt am Tisch vor den Leuten. In diesem Fall notiert man natürlich auch Sachen auf der Printversion. 

\medskip \emph{Clemens}: Gibt es in diesem Stadium auch schon andere Artefakte als Skizzen? Ich meine, verwendest du auch schon andere Materialien als Papier, z.B. Plastilin etc.?

\medskip \emph{Peter}: Das kommt auf das Projekt drauf an. Wenn du Designs hast, die davon leben, wo es z.B. auf die Ergonomie drauf ankommt, speziell bei Produkten die gut in der Hand liegen müssen, dann kommt das schon vor. Styropor eignet sich in solchen Fällen oft gut.

\medskip \emph{Clemens}: Aber das Skizzieren kommt schon immer vorher oder?

\medskip \emph{Peter}: Genau.

\medskip \emph{Clemens}: Sind Artefakte beim Designprozess wichtig? Ich meine Modelle, Skizzen, etc.

\medskip \emph{Peter}: Ja natürlich, äußerst wichtig. Du entwickelst ja Formen und Gebilde, die du sonst gar nicht beschreiben und dadurch auch nicht anderen kommunizieren könntest.

\medskip \emph{Clemens}: Wie sieht es mit Farben aus beim Skizzieren?

\medskip \emph{Peter}: Nun, beim skizzieren am Papier mache ich eigentlich nur Schattierungen, keine Farben. Diese füge ich dann später digital hinzu, hauptsächlich mit Photoshop. Es ist nämlich immer etwas schwierig beim Gespräch mit dem Kunden. Wenn man Farbe reinbringt, dann rutscht die Diskussion oft ab und der Kunde versteift sich auf die Farben, obwohl diese eigentlich nur exemplarisch sind und der Fokus der Diskussion eigentlich woanders liegen sollte.

\medskip \emph{Thomas}: Kannst du die Skizzen einfach abgeben, sodass der Kunde sich diese selbst ansehen kann oder ist es wichtig, dass du dabei stehst und eine Erklärung bietest?

\medskip \emph{Peter}: Ich persönlich erkläre immer meine Designs. Ob das auch anders rum funktionieren kann weiß ich jetzt nicht so recht. Vielleicht wenn man schon in einer weit fortgeschrittenen Phase des Designs ist und alle Beteiligten schon die selben konkreten Vorstellungen vom Resultat haben, könnte man eventuell technische Zeichnungen oder ähnliches auch kommentarlos abgeben.

\medskip \emph{Clemens}: Arbeitest du eigentlich auch im Team mit anderen Designern oder hauptsächlich nur alleine?

\medskip \emph{Peter}: Nun ja, ich hab schon viele verschiedene Situationen erlebt. Im Moment ist es so, dass ich Produktdesign alleine mache, aber ich habe auch oft schon mit andern Designern zusammengearbeitet. Meistens haben wir dann Screen-Sharing via Internet verwendet und zusätzlich noch telefoniert. Skype kann ja beides und hat sich als nützliches Tool zur Zusammenarbeit herausgestellt. Während der Besprechung skizziert gleich einer der beiden mit, z.B. auf Photoshop, der andere kann das live mitverfolgen und gemeinsam werden die Konzepte so ausgearbeitet. 

\medskip \emph{Thomas}: Könntest du dir vorstellen, dass zwei Designer gleichzeitig an der selben Skizze arbeiten?

\medskip \emph{Peter}: Hmm, meistens ist es schon so, dass jeder für sich skizziert und der andere entweder zuschaut und Inputs liefert oder die Konzepte nachher verglichen und elaboriert werden.

\medskip \emph{Clemens}: Bringt das parallele Arbeiten mehr Ideen hervor?

\medskip \emph{Peter}: Ja, natürlich.

\medskip \emph{Clemens}: Meinst du würden weniger Ideen produziert, wenn beide an der selben Skizze arbeiten würden?

\medskip \emph{Peter}: Das ist schwierig zu sagen. Vielleicht würden weniger grundlegend verschiedene Konzepte zustande gebracht werden, aber beim gemeinsamen Ausarbeiten werden die Ideen beider Designer sofort in einem Konzept vereint, das dann andererseits schon besser durchdacht ist. Diese Arbeitsform kann ich mir aber eigentlich nur in einer sehr frühen und groben Ideenfindungsphase vorstellen, denn gleichzeitig am selben zu zeichnen ist sicherlich sehr schwierig.

\medskip \emph{Clemens}: Es muss ja nicht direkt gleichzeitig passieren, sondern in Absprache und abwechselnd.

\medskip \emph{Peter}: Ja gut, wenn immer nur einer der beiden zeichnet, es dann weiter reicht und der nächste daran weiter zeichnet, so eine Arbeitsform kann ich mir durchaus gut vorstellen.

\medskip \emph{Thomas}: Du unterrichtest ja auch Darstellungstechnik an der Universität für angewandte Kunst in Wien. Wie sehen deine Vorlesungen aus, wie vermittelst du dein Wissen?

\medskip \emph{Peter}: Nun, ich zeige die Unterlagen am Beamer anhand von \acp{PDF}. Darin sind unterschiedliche Arbeitsschritte und Techniken von gewissen Designs zu sehen. Außerdem habe ich immer Hand-Outs mit Beispielen, die ich den Studenten gebe. Auf diesen können sie dann selbst Details erarbeiten, häufig auch in Teams. Generelle wäre es im Unterricht natürlich sehr praktisch wenn man direkt digital vorzeichnen könnte. Das habe ich so noch nie gemacht.

\medskip (\emph{Peter beginnt mit Scribbler auf verschiedenen Fenstern zu zeichnen})

\medskip \emph{Thomas}: Könntest du dir vorstellen, dass auch Studenten gleichzeitig mit dir mitarbeiten, sprich auch direkt am selben Screen zeichnen?

\medskip \emph{Peter}: Ja das wäre super, wenn ich das im Unterricht einsetzen könnte. Es wäre spitze wenn jeder die Möglichkeit hätte, mitzuarbeiten. Wobei ich nicht ganz sicher bin ob Darstellungstechnik das passende Unterrichtsfach für Scribbler ist, weil da geht es konkret um das Zeichnen und dann reichen Grobskizzen leider nicht aus, sondern man muss schon detaillierte Zeichnungen anfertigen. Anders sieht es da bei Designvisualisierungen aus, die auch in meinem Unterricht vorkommen. Da wäre es wirklich cool, die Studenten direkt einzubinden. Jeder könnte auch Stichworte dazuschreiben und ähnlich einem Brainstorming vernetzen. Das könnte ich mir gut vorstellen. 

\medskip \emph{Thomas}: Sind diese primitiven Zeichenmöglichkeiten ausreichend, oder fehlt dir da was?

\medskip \emph{Peter}: Nein, für Darstellungstechnik reichen diese Möglichkeiten bei weitem nicht aus. Man braucht da Dinge wie Strichstärke, gerade Linien, etc. Das sind Sachen, die Photoshop kann und die sind auch wirklich notwendig. Aber bei solchen groben Sachen, wie ich sie hier jetzt am Screen gezeichnet habe kann das schon reichen. Wichtig ist natürlich auch die Transparenz von Linien. Beim Skizzieren beginnt man ja mit ganz leichten Strichen, die das grobe Grundgerüst darstellen und zeichnet dann mit mehr Druckstärke drüber, sodass die Linien deutlicher werden und die Skizze konkreter wird.

\medskip \emph{Peter}: Kennt ihr Graphic-Recording?

\medskip \emph{Clemens, Thomas}: Nein.

\medskip \emph{Peter}: Das ist eine recht interessante Geschichte. Es handelt sich um Leute, die Meetings mit zeichnen. Das passiert analog auf einem Blatt Papier mit Stift. Nehmen wir an da sitzen mehrere Techniker und andere in ein Projekt verwickelte Personen, die Konzepte verbal besprechen und dann gibt es einen Zeichner, der, während die Leute ihre Ideen artikulieren, diese direkt zu Papier bringt und aufzeichnet. Darauf baut dann die Diskussion weiter auf und die Teilnehmer können gleich auf die Skizzen eingehen und sie weiter entwickeln oder verwerfen. Am Ende kann man anhand der Bilder nachvollziehen, worüber gesprochen worden ist. Das wäre sicherlich auch ein Gebiet, bei dem man Scribbler oder ähnliche Anwendungen zum Einsatz bringen könnte.

\medskip \emph{Clemens}: Bei uns am Institut gibt es öfter Meetings in der Bibliothek, in der mehrere Tische in U-Form stehen und zu einem großen Screen gerichtet sind, wo alle hinsehen können. Wenn jetzt z.B. einer ein Design für eine Webpage macht, kann er das allen zeigen, und für die anderen soll es dann möglich sein mittels Tablets auf das Design zu zeichnen und zu sagen das gefällt mir das nicht usw. Also das war die Grundidee, warum wir begonnen haben das Programm zu entwickeln. Und jetzt suchen wir eben auch Parallelen zu anderen Designsettings, um herauszufinden ob es auch dort Einsatz finden könnte. Du hast uns dazu schon ein wenig Input gegeben. Vielleicht könnten wir jetzt kurz die derzeitigen Features von Scribbler durchbesprechen. Du hast vielleicht vorher nicht bemerkt, bzw. wir habens dir noch nicht gesagt: Die Möglichkeit zu Radieren gibt es eigentlich auch - die hintere Taste am Stift dient dazu. Doch da das Programm vektorbasierend ist, radiert es nur den letzten Strich. Ist diese Funktion zu wenig für deinen Gebrauch?

\medskip \emph{Peter}: Also wenn ich wirklich an einem Produkt arbeite, um eine Form herauszuarbeiten, wäre es zu wenig ja. Aber um einfache Sachen, wie z.B Notizen einzufügen oder Ideen zu formulieren, funktioniert es schon. Man müsste aber vielleicht anfangen das Programm häufiger zu verwenden, um ein gutes Feedback abgeben zu können.

\medskip \emph{Clemens}: Ich habe bemerkt, dass du bis jetzt nur direkt auf die Bilder bzw. Fenster gezeichnet hast und Notizen auch nur direkt darauf geschrieben hast. Aber du könntest auch außerhalb - wo Platz ist - zeichnen. War das nur Zufall?

\medskip \emph{Peter}: Das war nur Zufall ja. Das ist ganz gut, dass man außerhalb zeichnen kann und somit Sachen verknüpfen kann. 

\smallskip \emph{(zeichnet Pfeile von einem Fenster zum anderen)}

\smallskip Ich bin ein Typ, der mir gerne Zusammenhänge und Strukturen schafft.
Ich überlege gerade, wo man jetzt wirklich verschiedene Vorschläge hat.

\smallskip \emph{(öffnet seine Unterlagen vom \ac{USB}-Stick)} 

\smallskip Ich schau jetzt einmal ob ich noch irgendwo etwas habe mit verschiedenen Entwürfen. 

\smallskip \emph{(öffnet eine Reihe von Bildern)}

\smallskip Das war auch so eine Entwurfsphase, wo es verschiedene Vorschläge gibt, wie man das umsetzen könnte. Wenn ich mir das jetzt überleg, dass das in einem Gespräch stattfindet. Da ist es eben darum gegangen verschiedene Formen zu finden, wie man den Inhalt transportieren kann. Das war zb. die erste Idee.

\smallskip \emph{(schreibt >>1. Idee<< auf ein Bild)}

\medskip \emph{Peter}: Was hier [in Scribbler] auch gut funktioniert, ist das Verdeutlichen von Ideen. Ich bin jemand, der irrsinnig gerne zeichnet zum Reden. Ich könnte so z.B. ein paar Punkte rausholen und einen Teil, der hier im Bereich unten schwer zu erkennen ist, noch einmal rauszeichnen.

\medskip \emph{Clemens}: Du zeichnest derzeit mit der Farbe Magenta. Du hast vorher gemeint für Grobskizzen reicht dir eine Farbe oder?

\medskip \emph{Peter}: Also wenn ich in der Strukturierungsphase bin, wäre es schon gut wenn ich mehrere Farben hätte.

\medskip \emph{Clemens}: Ok. Dazu haben wir eigentlich mehrere Stifte mit unterschiedlichen Farben angedacht.

\medskip \emph{Peter}: Wirklich wahr?

\smallskip \emph{(Peter greift zu einem anderen Stift und probiert ihn aus)}

\smallskip Wahnsinn. Ja, das finde ich super wenn man das so löst.

\medskip \emph{Thomas}: Wir haben uns gedacht wir halten uns daran, möglichst realitätsnahe zu bleiben. Beim Zeichnen auf Papier würdest du auch einen anderen Stift zur Hand nehmen.

\medskip \emph{Peter}: Verstehe. Was ich irrsinnig gerne verwende sind Untermenüs bzw. Popupmenüs. Das gibt es bei verschiedenen Programmen. Wenn ich z.B. auf die Stifttaste drücke geht ein Untermenü auf. So wie bei >>Autodesk Maya<< oder >>Sketchbook<<. Das sind Programme in denen ich z.B. immer zeichne. Und da hätte man im Untermenü auch die Möglichkeit auf eine Farbpalette, oder vielleicht auch 2-3 verschiedene Strichstärken. Dazu drück ich auf die Stifttaste, fahre mit dem Stift auf das Menü, lass wieder aus, das Menü ist wieder weg und ich habe die neue Einstellung. Beim Zeichnen ist das super; das ist etwas was mir z.B. in Photoshop abgeht.

\medskip \emph{Thomas}: Gut, da würde man dann weg von dem Ansatz mit mehreren Stiften gehen, weil es natürlich viel schneller ist.

\medskip \emph{Peter}: Ja, aber es funktioniert dann schon intuitiv. Man kann es echt gut benutzen. Wieviele Stifte kann man da nehmen?

\medskip \emph{Thomas}: Unendlich viele.

\medskip \emph{Peter}: Aha. Also das finde ich auch eine gute Sache mit unendlich vielen Farben.

\medskip \emph{Clemens}: Mit der Maus navigierst du nicht besonders viel. Machst du alles nur mit dem Stift?

\medskip \emph{Peter}: Wenn ich arbeite, arbeite ich ausschließlich mit dem Stift ja. Mit dem Stift und der Tastatur. Meine Arbeitssituation sieht immer so aus, dass ich das Tablet vor mir habe und eine Box das Tablet auf der Hinterseite leicht anhebt, sodass ich es gleich in der richtigen Schräge habe. Die Tastatur habe ich links daneben stehen, womit ich leicht Shortcuts erreichen kann.

\medskip \emph{Thomas}: Das heißt du zeichnest am Liebsten in der Schräge?

\medskip \emph{Peter}: Genau. Also leicht angehoben ja. So zeichne ich wesentlich entspannter. Ich habe zusätzlich ein A3 Tablet und dadurch werde ich gezwungen größere Bewegungen zu machen, was gut für meinen Rücken ist.

\medskip \emph{Clemens}: Um noch einmal auf die Farben zurückzukommen: Wenn du dir vorstellst es arbeiten 5 Personen gleichzeitig an einem Design, dann braucht man vielleicht eine Unterscheidung vom jeweilig Gezeichneten. Wir dachten daran, dass jeder eine eigene Farbe bekommen könnte. Glaubst du das wäre problematisch, weil du vorher auch angemerkt hast, dass Farben oft mehr aussagen?

\medskip \emph{Peter}: Das mit den Farben würde ich so sehen: Vom Einsatz her, würde ich sagen dass du damit keine Präsentationszeichnungen machen kannst [wo man mehrere Farben benötigt]. So eine Zeichnung zu machen ist echt nicht einfach. Dazu braucht man Zeit und Konzentration. Dass daran mehrere Leute arbeiten kann ich mir nur schwer vorstellen. Zur Besprechung und Ideenfindung kann ich mir es sehr gut vorstellen.

\medskip \emph{Thomas}: Ich habe gesehen, du würdest dir auch wünschen dass du einzelne Teile von Zeichnungen wo anders hinschieben könntest oder? Also dass du einen bestimmten Bereich ausschneiden und verschieben könntest. 

\medskip \emph{Peter}: Ja, das wäre vielleicht speziell bei der Ideenfindung, wo alles durcheinander steht, oder auch bei Skizzen in einem Brainstorming technischer Natur interessant. Weil meistens ist der zweite Schritt dann der, dass ich anfange die Ideen zu ordnen. Also es wäre gut, um den ersten Prozess der Ideenfindung nicht zu stoppen bzw. ihm eine Hürde zu geben, sondern gleich weiterarbeiten zu können. Und da ist es eben notwendig das Ganze in eine Ordnung zu bringen. Dann hätte ich wirklich eine Lösung wo alle gemeinsam arbeiten können.

\medskip \emph{Thomas}: Wie machst du das wenn du mit Papier arbeitest? Schneidest du Skizzen aus und klebst sie irgendwo neu auf?

\medskip \emph{Peter}: Nein. Also ich hab einen bestimmten Workflow wie ich an die Sache heran gehe: Wenn ich im Ideenfindungsprozess bin, mach ich zuerst einmal Cluster bzw. Assoziationsketten, wo ich aus dem Bauch heraus das Thema beschreibe, um einmal abzuchecken, was dazu in meinem Kopf ist. Danach nehme ich ein neues Blatt Papier her und beginne die verschiedenen Skizzen zu kategorisieren indem ich die Skizzen geordnet neu zeichne.

\medskip \emph{Thomas}: Wird das dann auch genauer?

\medskip \emph{Peter}: Das wird auch genauer ja. Anfangs habe ich einen Pool von Gedanken und Assoziationen und die kann ich dann her nehmen und nach verschiedenen Kategorien einteilen.

\medskip \emph{Clemens}: Wäre das ein Vorteil für dich, wenn du eine elektronische Unterstützung hast, wo du deine Zeichnungen sofort neu ordnen könntest?

\medskip \emph{Peter}: Jaja. Na das wäre super. Das ist zum Beispiel immer das Problem wenn du mit >>Illustrator<< oder >>Photoshop<< arbeitest; da ist das sehr umständlich. In dem Bereich könnte ich mir das ganz gut vorstellen. Vor allem wenn verschiedene Leute mitarbeiten, dann würde >>Illustrator<< usw. mit der Werkzeugleiste wahrscheinlich überhaupt nicht funktionieren. Das würde für die Ideenfindung super funktionieren, wenn wirklich 3-4 Leute auf einen Screen arbeiten und das auch weiterverwenden können.

\medskip \emph{Clemens}: Könntest du zusätzlich auch eine reine weiße Fläche zum Zeichnen brauchen? Also eine Whiteboard Funktion? Oder reicht es dir, auf Fenster zeichnen zu können?

\medskip \emph{Peter}: Ich finde es grundsätzlich gut, wenn sich jeder ein eigenes Fenster hernehmen und dann direkt auf die vorhandenen Designs zeichnen kann. Ein zusätzliches Whiteboard wäre aber auch super.

\medskip \emph{Thomas}: Ein weiterer interessanter Punkt für uns ist, wie man das Gezeichnete abspeichern könnte. Momentan gibt es dafür einen simplen Screenshot. Ist es aus deiner Sicht notwendig, etwas so abzuspeichern, damit man an einem späteren Zeitpunkt daran wieder weiter arbeiten kann, oder reicht dir ein Screenshot?

\medskip \emph{Peter}: Das ist schwer zu sagen. Dazu müsste ich das Programm in meinen Workflow einbinden. Natürlich wäre es eine tolle Sache, wenn man beim Öffnen wieder genau die selbe Ansicht hat, auf der man zuvor gearbeitet hat. Aber ich kann jetzt nicht sagen, ob das ein wahnsinns Vorteil wäre. Dazu müsste ich es ausprobieren.

\medskip \emph{Thomas}: Habe ich es richtig verstanden, dass du es gerne so hättest, dass wenn mehrere gleichzeitig zeichnen, jeder auf sein eigenes Fenster zeichnet und die jeweiligen Zeichnungen auf dem eigenen Fenster kleben bleiben? Momentan hängen alle Zeichnungen - egal von wem gezeichnet - nur auf dem aktivem Fenster. Sobald du das Fenster verschiebst, verschiebst du die Zeichnungen mit.

\medskip \emph{Peter}: \emph{(denkt nach)} 

\smallskip Weiß ich nicht. Also für mich funktioniert das derzeit von der Überlegung her recht gut. Aber ich müsste es erst länger ausprobieren, um auch die wirklichen Stärken zu finden.

\medskip \emph{Thomas}: Hättest du Lust das Programm im Unterricht oder in der Arbeit auszuprobieren?

\medskip \emph{Peter}: Zum Ausprobieren wäre es sicher toll. Interessant wäre es auch wenn man übers Netz auf einen Bildschirm zugreifen könnte. Weil ich sehr oft so arbeite, dass ich nicht im selben Raum sitze, mit den Leuten mit denen ich zusammenarbeite. Das wäre dann aber wahrscheinlich wieder ein eigenes Programm. Aber gerade für den Ideenfindungsprozess und Besprechungen wäre es sehr interessant.

\medskip \emph{Clemens}: Zum Schluss würden wir gerne noch deine Meinung zu zukünftig angedachten Features hören. 
Da wir mit elektronischen Artefakten arbeiten, die durch den gemeinsam genutzten Bildschirm weiter von einem entfernt sind, entsteht eine Hürde bei der Gestik. Deswegen planen wir eine Zeigefunktion zu integrieren. Glaubst du würde ein am Screen angezeigter Laserpointer oder - vielleicht abstrakter - eine neigbare, durch den Tabletstift steuerbare Hand bei dem Problem helfen?

\medskip \emph{Peter}: Würde ich toll finden ja. Speziell bei Präsentationen kommt es immer wieder zu Verwirrungen wenn jemand etwas zu einem Design anmerkt, weil die anderen nicht wissen von was genau gesprochen wird. Natürlich könnte man jedem einfach einen Laserpointer in die Hand drücken. Aber ich kann mir das auch gut vorstellen wenn man den Cursor z.B. durch eine Hand austauscht, oder zu jedem Cursor den Namen dazu schreiben kann. Man müsste dann eben Umschalten können zwischen der Zeichenfunktion und der Zeigefunktion.

\medskip \emph{Clemens}: Eine weitere Idee von uns wäre den Screen für andere Computer im Netzwerk zu öffnen, damit Benutzer z.B. ein Fenster mit eigenen Content von ihrem Notebookscreen auf den großen Bildschirm ziehen können, auf das dann jeder zeichnen kann. Das könnte von Vorteil sein, da die meisten ihre eigenen Unterlagen nur bei sich am Rechner haben und man somit die Informationen für alle leicht zugänglich machen kann. Denkst du wäre das ein brauchbarer Ansatz?

\medskip \emph{Peter}: Was mir dazu einfällt ist, dass man bei Präsentationen oft mehrseitige Dokumente hat. Interessant wäre es nun die Seiten wie im >>Acrobat Reader<< auf der Seite in einer Miniaturansicht anzuzeigen, damit dann jeder der auf ein bestimmtes Thema zugreifen möchte (und eine gute Präsentation hat eben die ganzen Themen drinnen), die Seite rausziehen kann.

\smallskip \emph{(denkt kurz nach)} 

\smallskip Im Prinzip kann man das natürlich auch jetzt schon über eine Dateienverwaltung machen. Aber das wäre vielleicht ein nettes Feature, weil die Leute dann aktiv bei der Präsentation mitarbeiten können. Und wenn man dann auf einer rausgezogenen Seite gezeichnet hat und sie wieder minimiert, dann wandern die Zeichnungen mit usw. So wie es jetzt schon mit dem aktiven Fenster funktioniert.

\medskip \emph{Thomas}: Stören dich eigentlich die nachträglichen Korrekturen, die bei den gezeichneten Strichen vorgenommen werden?

\smallskip \emph{Anmerkung: Um Rechenleistung zu sparen, werden Linien von Scribbler während dem Zeichnen als Verkettung mehrerer Geraden dargestellt und erst nachträglich nach dem Absetzen des Stifts zu runden Kurven umgerechnet.}

\medskip \emph{Peter}: Nein, überhaupt nicht. Natürlich würde es bei genauen Arbeiten nicht funktionieren, aber für die Ideenskizzen usw. ist es überhaupt kein Problem. 

\smallskip \emph{(probiert genaues Zeichnen aus)}

\smallskip Mit ein bisschen Übung bekommt man aber auch das hin.

\medskip \emph{Clemens, Thomas}: Danke, das waren unsere Fragen. Hast du sonst noch Fragen oder Anregungen?

\medskip \emph{Peter}: Nein, das hat mir jetzt echt Spaß gemacht. Es ist super - ich müsste nur anfangen damit zu arbeiten um genaueres sagen zu können. Aber ich hoffe ich habe euch helfen können. Für mich ist das auf jeden Fall eine Traumsituation, da ich jetzt schon Notizen zu Präsentationen mache. Nur das Problem ist eben jetzt, dass ich die Notizen auf Zettel mache und mir dazuschreiben muss zu welcher Zeichnung die Notizen gehören, damit ich sie nachträglich wieder zuordnen kann.

\medskip \emph{Thomas}: Wäre das Programm deiner Meinung nach schon reif genug, um es in einer deiner Arbeitssituationen ausprobieren zu können?

\medskip \emph{Peter}: \emph{(denkt nach)}
Ob ich es wirklich sofort einsetzen würde bei einer Präsentation, weiß ich jetzt nicht. Was den Unterricht anbelangt, müsste ich meine Arbeit umstellen und sie darauf anpassen. Wo ich mir den Einsatz gut vorstellen kann, ist im privaten Rahmen wo ich in Teams arbeite.

\medskip \emph{Clemens, Thomas}: Super. Vielen Dank.

\clearpage
\section{Interview mit zwei User Experience Designern}
\emph{Pi} und \emph{Zed} sind von Beruf User Experience Designer und arbeiten hauptsächlich für Unternehmen, die im Bereich Telekommunikation tätig sind. 

\medskip \emph{Clemens}: Bitte erzählt uns kurz was ihr macht und wie ihr arbeitet. In welchen Situationen macht ihr Gebrauch von Skizzen?

\medskip \emph{Pi}: wir skizzieren hauptsächlich auf Flipcharts beim Kunden selbst. Heute beispielsweise saßen wir zusammen in einem Meetingraum und haben Bedienungsabläufe am Flipchart gezeichnet. Das Kundenmeeting wird eigentlich oft zu einem Design Workshop auch wenn das so gar nicht geplant ist. Manchmal dreht man die Situation auch konkret so hin, dass es zu einem Design Workshop wird, weil sich der Kunde die Dinge dann besser vorstellen kann.

\medskip \emph{Clemens}: Zeichnen die Kunden auch selbst?

\medskip \emph{Pi}: Ja das kommt durchaus vor. Ich zeichne bestimmte Konzepte auf und Kunden schreiben oder zeichnen selbst etwas dazu.

\medskip \emph{Clemens}: Also braucht ihr das Zeichnen hauptsächlich vor Ort beim Kunden?

\medskip \emph{Zed}: Nein es gibt natürlich auch die Situationen in denen wir hier gemeinsam im Team skizzieren: auf Whiteboards, Blöcken oder auch auf tischgroßen Papieren.

\medskip \emph{Clemens}: Ok, wollt ihr noch kurz erklären, was ihr allgemein macht?

\medskip \emph{Zed}: Ja, wir sind beide Geschäftsführer der intuio User Experience Consulting GmbH, die wir gemeinsam im Jahre 2008 gegründet haben. Wir leben von Ablauf- und Interaktionsdesign, wo Sketches und Visualisierung von Abläufen und Zeichnungen eine große Rolle spielen, natürlich auch in Kollaboration mit Kollegen und Kunden. 

\medskip \emph{Clemens}: Super. Wir haben euch ja schon erzählt, was Scribbler ist. Vielleicht probiert ihr es einfach mal aus.

\medskip (\emph{Zed nimmt den Stift in die Hand und beginnt zu zeichnen})

\medskip \emph{Zed}: Also ich hatte leider noch nie so ein Wacom Tablett zum probieren. Ich bin da etwas ungeschickt... Ah, so funktioniert das also, wenn man den Stift aufsetzt wird sofort gezeichnet, ich dachte dann würde nur der Cursor bewegt und man zeichnet mit mehr Druck oder einer bestimmten Taste. 

\medskip (\emph{Übergibt den Stift an Pi, der auch probieren will})

\medskip \emph{Pi}: Aha, er reagiert schon wenn man in die Nähe kommt. Für mich wirkt die absolute Projektion des Tablettbereichs auf den Bildschirm nicht sehr intuitiv. Ich würde den Cursor lieber >>schieben<<, wie bei einer Maus.

\medskip \emph{Thomas}: Bei Bedarf kann man die Projektion auch relativ einstellen.

\medskip \emph{Pi}: Ah, ok. Wozu sind eigentlich diese Tasten hier gut? (Zeigt auf die zwei Tasten am Stift) 

\medskip \emph{Clemens}: Das sind die Funktionen Undo und Redo.

\medskip \emph{Zed}: Aah, ich dachte die wären zum zeichnen... Gibt es sonst noch Tastenfunktionen?

\medskip \emph{Clemens}: Ja, diese Tasten hier am Tablett sind mit Funktionen belegt. Man kann einen Screenshot machen, alle Linien löschen, ein Whiteboard einblenden und in den Mausmodus wechseln. Im Mausmodus zeichnet der Stift nicht, sondern kann als Maus verwendet werden. Außerdem ist ein Radierer am hinteren Ende des Stifts.

\medskip \emph{Zed}: Ah, sehr gut.

\medskip \emph{Pi}: Also das Radieren find ich weitaus besser als die Undo Funktion.

\medskip \emph{Zed}: Kannst du eigentlich schreiben? Ich mein, mit dem Stift Dinge auf den Screen schreiben. 

\medskip (\emph{Pi versucht zu schreiben})

\medskip \emph{Zed}: Nicht so gut oder?

\medskip (\emph{Pi hat Schwierigkeiten})

\medskip \emph{Clemens}: Warum tust du dich schwer beim Schreiben?

\medskip \emph{Pi}: Hmm, warum ist das so..? Moment... erstens sitze ich nicht gerade zum Tablett... Ich versuche jetzt, auf dieser Linie gerade zu schreiben. Ah, ungenau wird das,

\medskip \emph{Zed}: Ist die Übersetzung ungenau?

\medskip \emph{Pi}: Naja, wenn du nahe hinkommst reagiert der Cursor ja schon und diese leichte Verschiebung bevor man wirklich aufsetzt, stört mich. 

\medskip \emph{Zed}: Ja, daher würde ich lieber kontrollieren können, wann ich tatsächlich zu schreiben beginne. Darf ich mal probieren?

\medskip (\emph{Nimmt den Stift})

\medskip \emph{Zed}: Uah... Ja das ist wirklich schwierig. Ich komm nicht genau hin, wie ich mir das vorstelle. Wieso wird das so schief?

\medskip \emph{Thomas}: Das liegt daran, dass du schief zum Tablett sitzt.

\medskip \emph{Zed}: Ach deswegen. Ja es ist schwierig.

\medskip \emph{Clemens}: Also weil der Cursor nicht dort ist, wo man ihn erwarten würde.

\medskip \emph{Pi}: Genau.

\medskip \emph{Zed}: Mir fehlt die Übersetzung vom Tablett auf den Screen. Vielleicht wäre eine relative Projektion doch besser.

\medskip \emph{Clemens}: Ok, probieren wir es aus.

\medskip (\emph{Stellt die Projektion um})

\medskip \emph{Clemens}: Würde es helfen, wenn der Stift erst auf Druck reagieren würde?

\medskip \emph{Zed}: Ja, denke schon. Vielleicht wäre ein Tablet mit integriertem Screen besser, obwohl die auch diesen Hover Effekt haben.

\medskip \emph{Clemens}: Dass man sieht, wo man aufsetzt, sollte es erleichtern.

\medskip \emph{Zed}: Ja, du hast Recht. Die Übersetzung, oder anders gesagt, das Mapping fällt natürlich weg aber trotzdem würde der Hover Effekt stören.

\medskip (\emph{Clemens hat inzwischen die Projektion umgestellt})

\medskip \emph{Pi}: Aah, wesentlich besser! Probier mal.

\medskip (\emph{Gibt Zed den Stift})

\medskip \emph{Zed}: Ja, der Cursor wirkt nun ruhiger. Ah, schau her, die Linien werden ja nachträglich geglättet. Wenn ich schnell zeichne, wird die Linie erst kantiger. Das ist für schnelle Schreiben wahrscheinlich etwa ungünstig.

\medskip \emph{Thomas}: Das liegt daran, dass weniger Punkte registriert werden, wenn man den Stift schnell bewegt.

\medskip \emph{Pi}: Gut, es ist ja auch eher zum Zeichnen gedacht als zum Schreiben. Obwohl, beim Skizzieren muss man schon immer auch ein paar Notizen dazu machen.

\medskip \emph{Zed}: Ja das ist schon sehr wichtig. Also ich finde das Hauptproblem nach wie vor die Übersetzung, aber das ist bestimmt auch Übungssache. Das Problem hat man ja auch mit der Maus, aber die benutzt jeder schon seit Jahren und kann daher gut damit umgehen. Trotzdem fände ich die Wacom Tabletts mit integriertem Screen die bessere Lösung. 

\medskip (\emph{Pi zeichnet außerhalb des Browserfensters})

\medskip \emph{Zed}: Ah, jetzt sind wir außerhalb des Fensters. Das bedeutet die Zeichenfläche ist nicht Auf das Fenster beschränkt. Versuch mal ein wirkliches Interface Redesign zu skizzieren auf dieser Website, damit wir sehen, ob das Programm den eigentlichen Zweck erfüllt.

\medskip (\emph{Pi zeichnet})

\medskip \emph{Zed}: Ja man sieht, wenn man zu gewissen gezeichneten Dingen etwas dazuschreiben möchte und es ist nur wenig Platz vorhanden, dann wird's schwer. Zum Schreiben ist das leider zu ungenau.

\medskip \emph{Pi}: So, wenn ich jetzt eine Linie vom Ende der Website bis oben zum Anfang ziehen möchte, dann müsste ich gleichzeitig scrollen können.

\medskip \emph{Thomas}: Puh, das geht leider nicht.

\medskip (\emph{Pi zeichnet eine Linie, setzt ab, scrollt nach oben und verlängert dann die Linie})

\medskip (\emph{Pi versucht wieder zu schreiben})

\medskip \emph{Zed}: Schreiben ist nichts. Schreiben ist fast unmöglich.

\medskip (\emph{Pi versucht einen Textabsatz einzukreisen, es gelingt ihm jedoch nicht besonders gut})

\medskip \emph{Zed}: Warum geht das nicht, warum funktioniert das nicht so wie man das möchte?

\medskip \emph{Clemens}: Das Schreiben?

\medskip \emph{Zed}: Nein, das Einkringeln von bestimmten Dingen. Die Kreise werden so schief und verzogen.

\medskip \emph{Thomas}: Es ist schwierig, am Tablett zu zeichnen, während man auf das Display schaut, das muss man üben.

\medskip (\emph{Pi zeichnet inzwischen einen Strich und erreicht dabei mit dem Stift den Rand des Tabletts, am Screen geht der Strich jedoch nicht ganz bis zum Displayrand})

\medskip \emph{Pi}: Ah, schau, jetzt bin ich nach oben gefahren und jetzt steh ich hier an.

\medskip \emph{Thomas}: Ja das kommt durch die relative Projektion, die wir am Tablett eingestellt haben.

\medskip \emph{Pi}: Ach so.

\medskip \emph{Clemens}: Vielleicht sollten wir doch wieder umstellen.

\medskip \emph{Pi}: Nein das passt schon so.

\medskip \emph{Zed}: Können wir die Beschleunigung runter drehen?

\medskip \emph{Clemens}: Ja, klar.

\medskip (\emph{Verringert die Beschleunigung})

\medskip \emph{Zed}: Ok, das ist jetzt sehr langsam. (lacht)

\medskip \emph{Clemens}: Gut, ich stelle mal wieder um auf absolute Projektion.

\medskip \emph{Pi}: Ok.

\medskip (\emph{Stellt die Projektion um})

\medskip \emph{Pi}: Ja es ist einfach schwierig mit dieser Hardware. Die Umsetzung find ich super, aber durch die Eigenheiten der Hardware wird die Benutzbarkeit beeinträchtigt.

\medskip \emph{Clemens}: Was haltet ihr von der nachträglichen Glättung der Linien? Stört euch das?

\medskip \emph{Pi}: Also ich hab es gar nicht bemerkt.

\medskip \emph{Zed}: Ich finde es nett, es muss auch sein, glaub ich. Mich stört aber die Abtastfrequenz, weil wenn man schnell zeichnen möchte, dann funktioniert das nicht optimal.

\medskip \emph{Clemens}: Wenn wir das Programm auf einem stärkeren Rechner laufen lassen, geht es besser.

\medskip \emph{Zed}: Ah, ja.

\medskip \emph{Thomas}: Die eigentlichen Key-Features haben wir ja noch gar nicht gesehen. Ich spreche davon, dass Gezeichnetes an Fenstern und Inhalten haften bleibt. Das bedeutet, dass man Fenster verschieben und darin scrollen kann und die Zeichnungen automatisch mit wandern.

\medskip \emph{Zed}: Das Highlighting des aktiven Fensters, auf das gezeichnet wird, irritiert mich etwas. Weil es jedes mal kommt, wenn ich mit dem Stift zum Tablett hingehe. Hmmmm... das müsste glaub ich nicht immer da sein. Außer es ändert sich das aktive Fenster. 

\medskip (\emph{Wechselt zwischen Programmen und bemerkt, dass die Zeichnungen der inaktiven Programme nicht angezeigt werden})

\medskip \emph{Zed}: Wenn man umschaltet (zwischen den Programmen), dann ist die Linie außerhalb dieses Fensters auch weg, richtig?

\medskip \emph{Thomas}: Ja, genau.

\medskip \emph{Pi}: So, wenn ich wieder zum anderen Programm wechseln möchte, muss ich jetzt extra in den Mausmodus umschalten.

\medskip \emph{Clemens}: Naja, du kannst auch die Maus direkt benutzen.

\medskip \emph{Pi}: Aaah, es funktioniert beides.

\medskip \emph{Zed}: Aaah, das ist sinnvoll.

\medskip \emph{Pi}: Ja, mir hat die ganze Zeit so was wie Gestensteuerung gefehlt oder so. Aber jetzt wenn ich mit dem Stift zeichnen und mit der Maus steuern kann, dann ist das gut.

\medskip \emph{Pi}: Vielleicht ist die relative Projektion des Tabletts dann sinnvoll, wenn man zwei Eingabegeräte verwendet. Dann setze ich mit der Maus den Zeiger dorthin, wo ich ihn möchte und kann sofort dort los zeichnen, egal wo am Tablett sich mein Stift befindet.

\medskip \emph{Zed}: Aah, ja das ist eine gute Idee.

\medskip \emph{Pi}: Oder man könnte auch überhaupt zwei Tablets haben, das eine zur Steuerung mit Gesten und das andere zum Zeichnen.

\medskip \emph{Zed}: Da würde sich doch das Magic Pad von Apple anbieten.

\medskip (\emph{Zed beobachtet, wie Pi zeichnet und zwischen Fenstern wechselt})

\medskip \emph{Zed}: Das heißt es hängt alles an einem Fenster dran... hmmm, kannst du mal bitte das hintere Fenster aktivieren und es so verschieben, dass man die Fenster und Zeichnungen dahinter sieht?

\medskip \emph{Thomas}: Hmm, das geht so nicht. Es werden immer nur die Zeichnungen auf dem aktiven Fenster angezeigt.

\medskip \emph{Zed}: Ah, hmm. Ich glaub das irritiert mich, dass die Zeichnungen verschwinden. Vielleicht sollte man die ausgegraut anzeigen, so wie die inaktiven Fenster selbst.

\medskip \emph{Clemens}: Es besteht dann die Gefahr, dass ein Chaos entsteht, wenn zu viele Linien angezeigt werden.

\medskip \emph{Zed}: Ja stimmt, aber man könnte doch mit verschiedenen Alpha-Werten arbeiten, je nach dem, wie weit hinten sich ein Fenster befindet.

\medskip \emph{Thomas}: Oh, ja das klingt nach einer guten Idee.

\medskip \emph{Pi}: Man könnte dadurch den Kontext wahren und sehen was man bereits gemacht hat. 

\medskip \emph{Clemens}: Gut, wenn aber jetzt angenommen auf drei Fenstern gezeichnet wurde und die Zeichnungen ragen über die jeweiligen Fenster hinaus, wäre es dann nicht schwierig den Überblick zu behalten, welche Zeichnungen zu welchem Fenster gehören?

\medskip \emph{Pi}: Könnte man nicht auch jedem Fenster eine bestimmte Farbe geben, sodass z.B. alle Zeichnungen von Fenster A rot und alle Zeichnungen von Fenster B blau sind?

\medskip \emph{Thomas}: Da muss man aufpassen und bedenken, dass die Inhalte der Fenster ja nicht zwangsläufig weiß sein müssen, wie es bei vielen Webseiten der Fall ist. Es könnte dann passieren, dass die Zeichnungen zu wenig Kontrast zum Fenster haben.

\medskip \emph{Pi}: Ja, das stimmt.

\medskip \emph{Thomas}: In einem Testsetting hatten wir daher mal mehrere Stifte und jeder hatte eine bestimmte Farbe zugeordnet. Man konnte dann mit beliebigen Farben zeichnen.

\medskip \emph{Pi}: Ah, ja. Außerdem könnte man Farben auch nutzen um verschiedene Benutzer zu repräsentieren.

\medskip \emph{Thomas}: Genau.

\medskip \emph{Zed}: Wie sieht es eigentlich mit der Strichstärke aus? Kann man die anpassen?

\medskip \emph{Thomas}: Im Prototypen haben wir das nicht umgesetzt. Prinzipiell würde die Strichstärke über den Druck des Stifts auf dem Tablett geregelt. Photoshop kann das zum Beispiel.

\medskip \emph{Zed}: Ok. Das wäre sicher ein nettes Feature. 

\medskip (\emph{Denkt nach})

\medskip \emph{Zed}: Aber in der Praxis kann ich mir das schwer vorstellen (\emph{Anm.: den Einsatz von Scribbler}).

\medskip \emph{Thomas}: Warum?

\medskip \emph{Zed}: Weil es einfach noch so unrund läuft in der Bedienung. Beim kollaborativen Zusammenarbeiten mit dem Kunden glaub ich nicht dass es funktionieren würde. Der Kunde kann das nicht bedienen, allein der Umgang mit dem Tablett ist eher komplex. Hinzu kommt die notwendige Hardware, die ja auch erst ein mal angeschafft werden muss und transportiert werden muss. Man benötigt dann ja mehrere Tabletts. 

Ja das sind so die Gründe, die mir einfallen. Aber hauptsächlich liegt es an der unausgereiften Bedienung. Es müsste auch eine Möglichkeit geben, die Sachen abzuspeichern, vielleicht nicht nur ein Screenshot.

\medskip \emph{Thomas}: Ja, es gibt ein paar Überlegungen unsererseits, wie man das sinnvoll speichern könnte. Eine Variante sähe so aus, dass Scribbler sich merkt zu welchem Fenster eine Zeichnung gehört und würde anbieten diese zu laden, sobald erkannt wird, dass der User jenes Fenster geöffnet hat. Die zweite Variante würde einen Screenshot der Fenster machen und die Zeichnungen separat speichern. Der Screenshot könnte dann halb transparent über den Desktop gelegt werden, sodass der Benutzer darunter die ursprünglichen Fenster wieder so anordnen kann, wie sie hin gehören.

\medskip \emph{Zed}: Uh... das stelle ich mir schwierig vor. Da müsste ich echt alles wieder genau hinkriegen wie es war. Wenn es sich um komplexe Zeichnungen handelt ist das aufwändig und zeitintensiv. Total mühsam. Fraglich auch wie das dann aussieht wenn ich was in einer anderen Bildschirmauflösung öffne.

\medskip \emph{Thomas}: Stimmt, das hatten wir so noch nicht bedacht.

\medskip \emph{Pi}: Gut, die Vektorzeichnungen könnte man ja relativ einfach skalieren, wenn man die Bildschirmauflösung kennt.

\medskip \emph{Clemens}: Das Problem ist natürlich, dass wir keinen Einfluss auf andere Programme nehmen können. Daher war der Screenshot die einfachste und natürlichste Lösung. Trotzdem wollen wir noch andere Konzepte ausarbeiten. Es gäbe auch die Möglichkeit ein Photoshop File zu sichern, das verschiedene Layer mit den einzelnen Fenstern und Zeichnungen beinhaltet.

\medskip \emph{Zed}: Aber ich kann mir auch den praktischen Nutzen nicht so recht vorstellen. Zu sagen, ich würde das System wirklich irgendwo einsetzen... ich weiß nicht. Das Szenario fehlt mir. Beim Kunden fällt das nämlich komplett flach.

\medskip \emph{Thomas}: Glaubt ihr würde es besser funktionieren, wenn man Tabletts mit integrierten Displays einsetzen würde? Der Benutzer sieht ja dann worauf er zeichnet und diese Übersetzung von Tablett auf Beamer fiele flach.

\medskip \emph{Zed}: Da muss ich jetzt aber auch noch mal nachfragen, denn es gibt ja beispielsweise am iPad schon einige Apps, mit denen man skizzieren kann. Wo wäre dann der Mehrwert? Obwohl, ich glaub ich kenn keine kollaborativen Tools dieser Art.

\medskip \emph{Pi}: Aber mit dem Finger zu zeichnen ist Wahnsinn, das kannst du vergessen.

\medskip \emph{Zed}: Da gibt es doch auch so eigene Stifte, die man kaufen kann.

\medskip \emph{Clemens}: Leider funktionieren die nicht wirklich. Ich hatte mal die Gelegenheit einen auszuprobieren und muss sagen es war enttäuschend. Man muss viel zu fest drücken, das ist nicht wirklich ergonomisch beim Zeichnen.

\medskip \emph{Zed}: Ah, ok schade.

\medskip \emph{Pi}: Ich glaube die Touch-Technologie vom iPad ist einfach nicht geeignet für solche Dinge.

\medskip \emph{Clemens}: Liegt das Problem von Scribbler dann eigentlich eher bei der Hardware?

\medskip \emph{Pi}: Sehe ich schon so, ja.

\medskip \emph{Zed}: Na, ich muss auch sagen, ich sehe das Anwendungsszenario für mich einfach nicht.

\medskip \emph{Pi}: Gut, wir sind halt auch keine Grafiker, ich weiß nicht ob die der Sache nicht besser gegenüber stünden.

\medskip \emph{Zed}: Richtig.

\medskip \emph{Clemens}: Würdet ihr eine normale Flipchart-Applikation bevorzugen?

\medskip \emph{Pi}: Naja, beim Erstentwurf, wenn es noch nichts gibt, dann ist natürlich eine weiße Fläche das sinnvollste zum Sketchen. Aber sicher, es wäre natürlich schon fein, wenn man bereits gefertigte Wireframes beim Kunden zeigt und dort dann ad-hoc Notizen und Skizzen dazu machen könnte. Das wäre sicherlich ein nettes Anwendungsszenario.

\medskip \emph{Zed}: Wer notiert? Du oder der Kunde?

\medskip \emph{Pi}: Wäre beides denkbar, wobei ich mir nicht sicher bin ob was Gescheites herauskommt wenn der Kunde da rumkritzelt.

\medskip \emph{Zed}: Da kommt nur Blödsinn raus (\emph{lacht}).

\medskip \emph{Pi}: Ja bei diesen kollaborativen Anwendungen ist natürlich auch immer die Frage, wie geordnet dieses Zusammenarbeiten von statten geht. 

\medskip \emph{Clemens}: Wir haben auch eine Whiteboard-Funktion angedacht, sodass man einfach mal auf eine weiße Fläche zeichnen kann.

\medskip \emph{Zed}: Auf jeden Fall, ja. Das finde ich wichtig. Ich glaube sogar, dass das eher ein Anwendungsszenario produzieren könnte, denn Fenster darunter liegen zu haben und auf dem Desktop zu zeichnen, das ist glaub ich selten notwendig. Soll das ganze eigentlich auch online durchführbar sein oder nur im lokalen Netzwerk?

\medskip \emph{Clemens}: Eigentlich nur lokal, naja, nicht mal im Netzwerk, denn es hängen ja alle Tabletts am selben Rechner.

\medskip \emph{Zed}: Ah, ok. Also das kann ich mir dann wirklich nur eher auf so einem Tabletop-System vorstellen, das groß genug ist, dass mehrere Leute gleichzeitig arbeiten können. Ich denke das wäre die Einstiegsbarriere gering genug. Aber diese Setup hier, da bin ich ehrlich gesagt skeptisch.

\medskip \emph{Pi}: Cool wäre natürlich, wenn man das so mit einer \ac{GUI}-Widget-Library verbinden könnte. Dass man beispielsweise \ac{GUI}-Elemente grob zeichnen kann und das System erkennt, sagen wir mal eine DropDown Liste und ersetzt das gezeichnete mit einer schönen Vektorgrafik. Dann könnte man sehr schnell und einfach \acp{GUI} aufzeichnen und hätte dann schon schön elaborierte Entwürfe.

\medskip \emph{Zed}: Das ist dann eigentlich Mustererkennung.

\medskip \emph{Pi}: Genau, im Prinzip kannst du dir dann deine Applikation zusammen zeichnen. So was würde ich schon cool finden.

\medskip \emph{Clemens}: Und so wie das System ist, ohne die Mustererkennung meinst du geht das nicht?

\medskip \emph{Pi}: Naja, es sieht dann halt so aus (\emph{zeigt auf eine schlecht gezeichnete DropDown Liste}). Ich denke dabei an die Präsentation beim Kunden. Da muss das dann schon gut aussehen. Oder auch anders: Wir hatten heute erst ein Meeting bei dem wir Entwürfe auf ein Whiteboard gezeichnet haben und einer muss danach hergehen, die Skizzen abfotografieren und schöne Wireframes daraus machen. Diesen Arbeitsschritt würde man sich dann natürlich sparen.

\medskip \emph{Zed}: Ich habe den Eindruck, dass das so nur von Experten benutzt werden kann, weil die Technik da noch eine große Hürde darstellt. Aber auch beim Erkennen von \ac{GUI} Elementen muss ich wissen, wie muss ich das Zeichnen, damit es erkannt wird. Außerdem muss ich alle Elemente auswendig kennen, denn irgendwo alle auf einen Blick hab ich ja nicht. Generell stelle ich den Nutzen vom System, so wie es jetzt ist, in Frage. Die Einstiegshürden, verbunden mit den Hardwarekosten... ich bin nach wie vor skeptisch.

\medskip \emph{Thomas}: Was wäre notwendig, um es für dich nutzbar zu machen?

\medskip \emph{Zed}: Dieser komischer Hovereffekt des Stifts am Tablett bereitet mir Schwierigkeiten. Das müsste man auf jeden Fall irgendwie lösen. Sodass man das Gefühl bekommt, dass man wirklich exakt arbeiten kann mit dem Ding. Das fehlt mir. Schön wäre natürlich auch, wenn man die Übersetzung von Tablett auf Screen lösen könnte, wobei das natürlich nur dann geht, wenn der Screen im Tablett integriert ist. Auch cool wäre, wenn man über das Internet zusammenarbeiten könnte, denn so ist man an dieses Setting im Raum gebunden, nicht mal über ein lokales Netzwerk hat man da irgendwelche Freiheiten.

\medskip \emph{Pi}: Gut, es ist offensichtlich: Mit dem Stift kann ich immer noch nicht besser zeichnen, als mit der Maus. Da muss die Technologie verbessert werden. Klar ist das Übungssache, aber ein Bleistift und ein Blatt Papier sind einfach nach wie vor einfacher zu handhaben.

\medskip \emph{Clemens}: Wir haben noch eine Idee, wie man dieses Übersetzungsproblem von Tablett auf Screen eventuell lindern könnte. Man weiß ja nie so recht, wo man sich gerade befindet am Screen. Bei der Literaturrecherche sind wir auf ein System gestoßen, das die Hände der Benutzer beim Zeichnen filmt und halbtransparent am Screen darstellt, sodass man sieht, wo sich die eigene Hand befindet.

\medskip \emph{Pi, Zed}: Aaah, gute Idee.

\medskip \emph{Clemens}: Also könntet ihr euch vorstellen, dass das helfen würde.

\medskip \emph{Pi, Zed}: Ja.

\medskip \emph{Pi}: Aber ist das nicht eigentlich ein Workaround für etwas, das eigentlich umgekehrt gelöst werden sollte? Ich meine, man projiziert die Hände des Users auf den Screen aber eigentlich möchte er den Screen bei seiner Hand haben, so wie bei einem Blatt Papier. Das ist eine technische Lösung weil es so einfacher geht.

\medskip \emph{Clemens}: Der Vorteil wäre natürlich, dass zwei Personen am selben Ort zeichnen könnten, ohne dass sich die Hände in die Quere kommen.

\medskip \emph{Zed}: Ah, gut, das ist aber kein Problem, das ich gelöst haben will, glaub ich.

\medskip \emph{Clemens}: Ihr habt vorhin angesprochen, dass Kunden gerne vorne am gebeamten Bild Sachen zeigen. Was würdet ihr von einer Zeigefunktion halten? Z.B. könnte man einen digitalen Laserpointer implementieren oder ähnliches?

\medskip \emph{Zed}: Ich glaube die Tatsache, dass man aufstehen und nach vorne gehen muss, um etwas zu zeigen ist schon wichtig und gut so. Von der Ferne mit einem digitalen Laserpointer drauf zu zeigen ist einfach etwas anderes. 

\medskip \emph{Clemens}: Und wenn man sich überlegt, dass die Hand eingeblendet wird am Screen, damit man mit dem Finger über die Distanz zeigen kann?

\medskip \emph{Zed}: Ich glaube nicht, dass das zum Zeigen gut wäre, aber es könnte das Übersetzungsproblem beim Zeichnen lindern oder sogar lösen.

\medskip \emph{Pi}: Ja das könnte schon funktionieren. Es ist natürlich fraglich, was da wieder für ein Overhead zusammenkommt. Einerseits ist es sicherlich viel Rechenaufwand, mehrere Kamerabilder live, halbtransparent übereinander zu legen. Andererseits muss man auch schauen, wie viel Clutter da am Screen entsteht und ob das ganze dann noch übersichtlich ist. Fraglich. Müsste man testen. Die Hände kommen ja höchstwahrscheinlich auch alle von der selben Seite, sprich rechts unten und dann liegen alle genau übereinander. Ich weiß nicht.

\medskip \emph{Clemens}: Ja, wiederum fraglich wie viele Personen dann wirklich gleichzeitig zeichnen.

\medskip \emph{Pi}: Stimmt.

\medskip \emph{Clemens}: Gut, das wars dann eigenlich. Danke, dass ihr euch für uns Zeit genommen habt.

\medskip \emph{Pi, Zed}: Gerne.

\medskip \emph{Thomas}: Danke.

















