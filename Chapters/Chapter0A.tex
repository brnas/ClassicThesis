%********************************************************************
% Appendix
%*******************************************************
\chapter{Gespr{\"a}che mit Designern}
Der folgende Abschnitt umfasst Transkripte, von uns geführten Interviews mit Designern aus verschiedenen Designsparten. Sie bieten Einblicke in Arbeitssituationen und Arbeitsweisen in den Bereichen \emph{Produktdesign}, \emph{Interaktionsdesign} und \emph{Webdesign} bzw. Erfahrungen und Anmerkungen zu Scribbler.

\medskip \emph{Anmerkung: Die Abschrift und alle dazugehörigen Materialen, wurden von den jeweiligen Interviewpartnern gestattet. Unter Rücksicht auf Anonymität werden lediglich Vornamen oder Initialen zur Unterscheidung verwendet. Von Thomas Nägele und Clemens Sagmeister werden stets die Vornamen verwendet.}

\section{Interview mit einem Produktdesigner}
\emph{Peter} ist von Beruf Produktdesigner und arbeitet zusätzlich als Professor an der Universität für angewandte Kunst Wien. 

\par...
\par \medskip \emph{Clemens}: Bei uns am Institut gibt es öfter Meetings in der Bibliothek, in der mehrere Tische in U-Form stehen und zu einem großen Screen gerichtet sind, wo alle hinsehen können. Wenn jetzt z.B. einer ein Design für eine Webpage macht, kann er das allen zeigen, und für die anderen soll es dann möglich sein mittels Tablets auf das Design zu zeichnen und zu sagen das gefällt mir das nicht usw. Also das war die Grundidee, warum wir begonnen haben das Programm zu entwickeln. Und jetzt suchen wir eben auch Parallelen zu anderen Designsettings, um herauszufinden ob es auch dort Einsatz finden könnte. Du hast uns dazu schon ein wenig Input gegeben. Vielleicht könnten wir jetzt kurz die derzeitigen Features von Scribbler durchbesprechen. Du hast vielleicht vorher nicht bemerkt, bzw. wir habens dir noch nicht gesagt: Die Möglichkeit zu Radieren gibt es eigentlich auch - die hintere Taste am Stift dient dazu. Doch da das Programm vektorbasierend ist, radiert es nur den letzten Strich. Ist diese Funktion zu wenig für deinen Gebrauch?

\medskip \emph{Peter}: Also wenn ich wirklich an einem Produkt arbeite, um eine Form herauszuarbeiten, wäre es zu wenig ja. Aber um einfache Sachen, wie z.B Notizen einzufügen oder Ideen zu formulieren, funktioniert es schon. Man müsste aber vielleicht anfangen das Programm häufiger zu verwenden, um ein gutes Feedback abgeben zu können.

\medskip \emph{Clemens}: Ich habe bemerkt, dass du bis jetzt nur direkt auf die Bilder bzw. Fenster gezeichnet hast und Notizen auch nur direkt darauf geschrieben hast. Aber du könntest auch außerhalb - wo Platz ist - zeichnen. War das nur Zufall?

\medskip \emph{Peter}: Das war nur Zufall ja. Das ist ganz gut, dass man außerhalb zeichnen kann und somit Sachen verknüpfen kann. 

\smallskip \emph{(zeichnet Pfeile von einem Fenster zum anderen)}

\smallskip Ich bin ein Typ, der mir gerne Zusammenhänge und Strukturen schafft.
Ich überlege gerade, wo man jetzt wirklich verschiedene Vorschläge hat.

\smallskip \emph{(öffnet seine Unterlagen vom USB-Stick)} 

\smallskip Ich schau jetzt einmal ob ich noch irgendwo etwas habe mit verschiedenen Entwürfen. 

\smallskip \emph{(Öffnet eine Reihe an Bildern)}

\smallskip Das war auch so eine Entwurfsphase, wo es verschiedene Vorschläge gibt, wie man das umsetzen könnte. Wenn ich mir das jetzt überleg, dass das in einem Gespräch stattfindet. Da ist es eben darum gegangen verschiedene Formen zu finden, wie man den Inhalt transportieren kann. Das war zb. die erste Idee.

\smallskip \emph{(schreibt >>1. Idee<< auf ein Bild)}

\smallskip --- Peter zeichnet auf Bilder; für bessere Beschreibung Video anschauen. ---

\medskip \emph{Peter}: Was hier [in Scribbler] auch gut funktioniert, ist das Verdeutlichen von Ideen. Ich bin jemand, der irrsinnig gerne zeichnet zum Reden. Ich könnte so z.B. ein paar Punkte rausholen und einen Teil, der hier im Bereich unten schwer zum erkennen ist, noch einmal rauszeichnen.

\medskip \emph{Clemens}: Du zeichnest derzeit mit der Farbe Magenta. Du hast vorher gemeint für Grobskizzen reicht dir eine Farbe oder?

\medskip \emph{Peter}: Also wenn ich in der Strukturierungsphase bin, wäre es schon gut wenn ich mehrere Farben hätte.

\medskip \emph{Clemens}: Ok. Dazu haben wir eigentlich mehrere Stifte mit unterschiedlichen Farben angedacht.

\medskip \emph{Peter}: Wirklich wahr?

\smallskip \emph{(Peter greift zu einem anderen Stift und probiert ihn aus)}

\smallskip Wahnsinn. Ja, das finde ich super wenn man das so löst.

\medskip \emph{Thomas}: Wir haben uns gedacht wir halten uns daran, möglichst realitätsnahe zu bleiben. Beim Zeichnen auf Papier würdest du auch einen anderen Stift zur Hand nehmen.

\medskip \emph{Peter}: Verstehe. Was ich irrsinnig gerne verwende sind Untermenüs bzw. Popupmenüs. Das gibt es bei verschiedenen Programmen. Wenn ich z.B. auf die Stifttaste drücke geht ein Untermenü auf. So wie bei >>Autodesk Maya<< oder >>Sketchbook<<. Das sind Programme in denen ich z.B. immer zeichne. Und da hätte man im Untermenü auch die Möglichkeit auf eine Farbpalette, oder vielleicht auch 2-3 verschiedene Strichstärken. Dazu drück ich auf die Stifttaste, fahre mit dem Stift auf das Menü, lass wieder aus, das Menü ist wieder weg und ich habe die neue Einstellung. Beim Zeichnen ist das super; das ist etwas was mir z.B. in Photoshop abgeht.

\medskip \emph{Thomas}: Gut, da würde man dann weg von dem Ansatz mit mehreren Stiften gehen, weil es natürlich viel schneller ist.

\medskip \emph{Peter}: Ja, aber es funktioniert dann schon intuitiv. Man kann es echt gut benutzen. Wieviele Stifte kann man da nehmen?

\medskip \emph{Thomas}: Unendlich viele.

\medskip \emph{Peter}: Aha. Also das finde ich auch eine gute Sache mit unendlich vielen Farben.

\medskip \emph{Clemens}: Mit der Maus navigierst du nicht besonders viel. Machst du alles nur mit dem Stift?

\medskip \emph{Peter}: Wenn ich arbeite, arbeite ich ausschließlich mit dem Stift ja. Mit dem Stift und der Tastatur. Meine Arbeitssituation sieht immer so aus, dass ich das Tablet vor mir habe und eine Box das Tablet auf der Hinterseite leicht anhebt, sodass ich es gleich in der richtigen Schräge habe. Die Tastatur habe ich links daneben stehen, womit ich leicht Shortcuts erreichen kann.

\medskip \emph{Thomas}: Das heißt du zeichnest am Liebsten in der Schräge?

\medskip \emph{Peter}: Genau. Also leicht angehoben ja. So zeichne ich wesentlich entspannter. Ich habe zusätzlich ein A3 Tablet und dadurch werde ich gezwungen größere Bewegungen zu machen, was gut für meinen Rücken ist.

\medskip \emph{Clemens}: Um noch einmal auf die Farben zurückzukommen: Wenn du dir vorstellst es arbeiten 5 Personen gleichzeitig an einem Design, dann braucht man vielleicht eine Unterscheidung vom jeweilig Gezeichneten. Wir dachten daran, dass jeder eine eigene Farbe bekommen könnte. Glaubst du das wäre problematisch, weil du vorher auch angemerkt hast, dass Farben oft mehr aussagen?

\medskip \emph{Peter}: Das mit den Farben würde ich so sehen: Vom Einsatz her, würde ich sagen dass du damit keine Präsentationszeichnungen machen kannst [wo man mehrere Farben benötigt]. So eine Zeichnung zu machen ist echt nicht einfach. Dazu braucht man Zeit und Konzentration. Dass daran mehrere Leute arbeiten kann ich mir nur schwer vorstellen. Zur Besprechung und Ideenfindung kann ich mir es sehr gut vorstellen.

\medskip \emph{Thomas}: Ich habe gesehen, du würdest dir auch wünschen dass du einzelne Teile von Zeichnungen wo anders hinschieben könntest oder? Also dass du einen bestimmten Bereich ausschneiden und verschieben könntest. 

\medskip \emph{Peter}: Ja, das wäre vielleicht speziell bei der Ideenfindung, wo alles durcheinander steht, oder auch bei Skizzen in einem Brainstorming technischer Natur interessant. Weil meistens ist der zweite Schritt dann der, dass ich anfange die Ideen zu ordnen. Also es wäre gut, um den ersten Prozess der Ideenfindung nicht zu stoppen bzw. ihm eine Hürde zu geben, sondern gleich weiterarbeiten zu können. Und da ist es eben notwendig das Ganze in eine Ordnung zu bringen. Dann hätte ich wirklich eine Lösung wo alle gemeinsam arbeiten können.

\medskip \emph{Thomas}: Wie machst du das wenn du mit Papier arbeitest? Schneidest du Skizzen aus und klebst sie irgendwo neu auf?

\medskip \emph{Peter}: Nein. Also ich hab einen bestimmten Workflow wie ich an die Sache heran gehe: Wenn ich im Ideenfindungsprozess bin, mach ich zuerst einmal Cluster bzw. Assoziationsketten, wo ich aus dem Bauch heraus das Thema beschreibe, um einmal abzuchecken, was dazu in meinem Kopf ist. Danach nehme ich ein neues Blatt Papier her und beginne die verschiedenen Skizzen zu kategorisieren indem ich die Skizzen geordnet neu zeichne.

\medskip \emph{Thomas}: Wird das dann auch genauer?

\medskip \emph{Peter}: Das wird auch genauer ja. Anfangs habe ich einen Pool von Gedanken und Assoziationen und die kann ich dann her nehmen und nach verschiedenen Kategorien einteilen.

\medskip \emph{Clemens}: Wäre das ein Vorteil für dich, wenn du eine elektronische Unterstützung hast, wo du deine Zeichnungen sofort neu ordnen könntest?

\medskip \emph{Peter}: Jaja. Na das wäre super. Das ist zum Beispiel immer das Problem wenn du mit >>Illustrator<< oder >>Photoshop<< arbeitest; da ist das sehr umständlich. In dem Bereich könnte ich mir das ganz gut vorstellen. Vor allem wenn verschiedene Leute mitarbeiten, dann würde >>Illustrator<< usw. mit der Werkzeugleiste wahrscheinlich überhaupt nicht funktionieren. Das würde für die Ideenfindung super funktionieren, wenn wirklich 3-4 Leute auf einen Screen arbeiten und das auch weiterverwenden können.

\medskip \emph{Clemens}: Könntest du zusätzlich auch eine reine weiße Fläche zum Zeichnen brauchen? Also eine Whiteboard Funktion? Oder reicht es dir, auf Fenster zeichnen zu können?

\medskip \emph{Peter}: Ich finde es grundsätzlich gut, wenn sich jeder ein eigenes Fenster hernehmen und dann direkt auf die vorhandenen Designs zeichnen kann. Ein zusätzliches Whiteboard wäre aber auch super.

\medskip \emph{Thomas}: Ein weiterer interessanter Punkt für uns ist, wie man das Gezeichnete abspeichern könnte. Momentan gibt es dafür einen simplen Screenshot. Ist es aus deiner Sicht notwendig, etwas so abzuspeichern, damit man an einem späteren Zeitpunkt daran wieder weiter arbeiten kann, oder reicht dir ein Screenshot?

\medskip \emph{Peter}: Das ist schwer zu sagen. Dazu müsste ich das Programm in meinen Workflow einbinden. Natürlich wäre es eine tolle Sache, wenn man beim Öffnen wieder genau die selbe Ansicht hat, auf der man zuvor gearbeitet hat. Aber ich kann jetzt nicht sagen, ob das ein wahnsinns Vorteil wäre. Dazu müsste ich es ausprobieren.

\medskip \emph{Thomas}: Habe ich es richtig verstanden, dass du es gerne so hättest, dass wenn mehrere gleichzeitig zeichnen, jeder auf sein eigenes Fenster zeichnet und die jeweiligen Zeichnungen auf dem eigenen Fenster kleben bleiben? Momentan hängen alle Zeichnungen - egal von wem gezeichnet - nur auf dem aktivem Fenster. Sobald du das Fenster verschiebst, verschiebst du die Zeichnungen mit.

\medskip \emph{Peter}: \emph{(denkt nach)} 

\smallskip Weiß ich nicht. Also für mich funktioniert das derzeit von der Überlegung her recht gut. Aber ich müsste es erst länger ausprobieren, um auch die wirklichen Stärken zu finden.

\medskip \emph{Thomas}: Hättest du Lust das Programm im Unterricht oder in der Arbeit auszuprobieren?

\medskip \emph{Peter}: Zum Ausprobieren wäre es sicher toll. Interessant wäre es auch wenn man übers Netz auf einen Bildschirm zugreifen könnte. Weil ich sehr oft so arbeite, dass ich nicht im selben Raum sitze, mit den Leuten mit denen ich zusammenarbeite. Das wäre dann aber wahrscheinlich wieder ein eigenes Programm. Aber gerade für den Ideenfindungsprozess und Besprechungen wäre es sehr interessant.

\medskip \emph{Clemens}: Zum Schluss würden wir gerne noch deine Meinung zu zukünftig angedachten Features hören. 
Da wir mit elektronischen Artefakten arbeiten, die durch den gemeinsam genutzten Bildschirm weiter von einem entfernt sind, entsteht eine Hürde bei der Gestik. Deswegen planen wir eine Zeigefunktion zu integrieren. Glaubst du würde ein am Screen angezeigter Laserpointer oder - vielleicht abstrakter - eine neigbare, durch den Tabletstift steuerbare Hand bei dem Problem helfen?

\medskip \emph{Peter}: Würde ich toll finden ja. Speziell bei Präsentationen kommt es immer wieder zu Verwirrungen wenn jemand etwas zu einem Design anmerkt, weil die anderen nicht wissen von was genau gesprochen wird. Natürlich könnte man jedem einfach einen Laserpointer in die Hand drücken. Aber ich kann mir das auch gut vorstellen wenn man den Cursor z.B. durch eine Hand austauscht, oder zu jedem Cursor den Namen dazu schreiben kann. Man müsste dann eben Umschalten können zwischen der Zeichenfunktion und der Zeigefunktion.

\medskip \emph{Clemens}: Eine weitere Idee von uns wäre den Screen für andere Computer im Netzwerk zu öffnen, damit Benutzer z.B. ein Fenster mit eigenen Content von ihrem Notebookscreen auf den großen Bildschirm ziehen können, auf das dann jeder zeichnen kann. Das könnte von Vorteil sein, da die meisten ihre eigenen Unterlagen nur bei sich am Rechner haben und man somit die Informationen für alle leicht zugänglich machen kann. Denkst du wäre das ein brauchbarer Ansatz?

\medskip \emph{Peter}: Was mir dazu einfällt ist, dass man bei Präsentationen oft mehrseitige Dokumente hat. Interessant wäre es nun die Seiten wie im >>Acrobat Reader<< auf der Seite in einer Miniaturansicht anzuzeigen, damit dann jeder der auf ein bestimmtes Thema zugreifen möchte (und eine gute Präsentation hat eben die ganzen Themen drinnen), die Seite rausziehen kann.

\smallskip \emph{(denkt kurz nach)} 

\smallskip Im Prinzip kann man das natürlich auch jetzt schon über eine Dateienverwaltung machen. Aber das wäre vielleicht ein nettes Feature, weil die Leute dann aktiv bei der Präsentation mitarbeiten können. Und wenn man dann auf einer rausgezogenen Seite gezeichnet hat und sie wieder minimiert, dann wandern die Zeichnungen mit usw. So wie es jetzt schon mit dem aktiven Fenster funktioniert.

\medskip \emph{Thomas}: Stören dich eigentlich die nachträglichen Korrekturen, die bei den gezeichneten Strichen vorgenommen werden?

\smallskip \emph{Anmerkung: Da Scribbler auf Vektorgrafiken basiert, werden die original gezeichneten Striche nachträglich an Vektorenlinien angeglichen.}

\medskip \emph{Peter}: Nein, überhaupt nicht. Natürlich würde es bei genauen Arbeiten nicht funktionieren, aber für die Ideenskizzen usw. ist es überhaupt kein Problem. 

\smallskip \emph{(probiert genaues Zeichnen aus)}

\smallskip Mit ein bisschen Übung bekommt man aber auch das hin.

\medskip \emph{Clemens, Thomas}: Danke, das waren unsere Fragen. Hast du sonst noch Fragen oder Anregungen?

\medskip \emph{Peter}: Nein, das hat mir jetzt echt Spaß gemacht. Es ist super - ich müsste nur anfangen damit zu arbeiten um genaueres sagen zu können. Aber ich hoffe ich habe euch helfen können. Für mich ist das auf jeden Fall eine Traumsituation, da ich jetzt schon Notizen zu Präsentationen mache. Nur das Problem ist eben jetzt, dass ich die Notizen auf Zettel mache und mir dazuschreiben muss zu welcher Zeichnung die Notizen gehören, damit ich sie nachträglich wieder zuordnen kann.

\medskip \emph{Thomas}: Wäre das Programm deiner Meinung nach schon reif genug, um es in einer deiner Arbeitssituationen ausprobieren zu können?

\medskip \emph{Peter}: \emph{(denkt nach)}
Ob ich es wirklich sofort einsetzen würde bei einer Präsentation, weiß ich jetzt nicht. Was den Unterricht anbelangt, müsste ich meine Arbeit umstellen und sie darauf anpassen. Wo ich mir den Einsatz gut vorstellen kann, ist im privaten Rahmen wo ich in Teams arbeite.

\medskip \emph{Clemens, Thomas}: Super. Vielen Dank.

\clearpage
\section{Interview mit zwei User Experience Engineers}
\emph{Thomas, P.} und \emph{Thomas, Z.} sind von Beruf User Experience Engineers ... 