%*******************************************************
% Abstract
%*******************************************************
%\renewcommand{\abstractname}{Abstract}
\pdfbookmark[1]{Abstract}{Abstract}
\begingroup
\let\clearpage\relax
\let\cleardoublepage\relax
\let\cleardoublepage\relax

\chapter*{Abstract}
When working together on a project, people share their documents, materials and tools. Real life objects allow users to comunicate thoughts, ideas and concepts very easily in short time. There has been some effort to recreate a similar experience when using digital media for collaborative work. Unfortunately, existing solutions still do not offer an efficient and transparent interface between real-life and digital artifacts which is the reason why people still prefer conventional methods for collaboration. \\
The following paper describes possible implementations to use traditional design methods on digital artefacts in theory and in practice. \scribbler is one representative of a collaborative system which uses \emph{sketching}, the most important design method, to find this missing link.

\vfill

\pdfbookmark[1]{Kurzfassung}{Kurzfassung}
\chapter*{Kurzfassung}
Kollaboratives Arbeiten bedeutet das Teilen von Materialien, Dokumenten und Werkzeugen. Herkömmliche, greifbare Medien schaffen eine Kommunikationsebene zwischen den Personen, auf der es möglich ist, Gedanken, Ideen und Konzepte rasch und zugänglich aufzubereiten. In der Vergangenheit sind schon einige Versuche unternommen worden, diese Art des Arbeitens auf digitalen Medien umzusetzen. So gibt es bereits Ansätze, die die Zusammenarbeit auf einem gemeinsamen großen Display ermöglichen sollen. Jedoch stößt man bei der Arbeit mit diesen kollaborativen Systemen immer wieder auf Barrieren, da es bisher noch nicht gelungen ist, eine effiziente und transparente Schnittstelle zwischen analogen und digitalen Objekten zu schaffen.\\
Anhand theoretischer und praktischer Ansätze beschreibt die folgende Arbeit die Unterstützung traditioneller Arbeitsweisen von Designern durch digitale Objekte. Das kollaborative System \scribbler setzt auf die wohl wichtigste Designmethode, das \emph{Skizzieren}, und versucht eine barrierefreie Schnittstelle zwischen realen und digitalen Objekten zu schaffen.

\endgroup			

\vfill