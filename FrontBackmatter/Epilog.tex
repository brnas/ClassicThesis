\cleardoublepage
%\pdfbookmark[-1]{Epilog}{Epilog}
\phantomsection
\addtocontents{toc}{\protect\vspace{\beforebibskip}} % to have the bib a bit from the rest in the toc
\addcontentsline{toc}{chapter}{\tocEntry{Epilog}}
\begin{quote}
	\begin{flushright}{\slshape    
	    Never doubt that a small group of thoughtful, committed people can change the world.\\
		Indeed, it is the only thing that ever has.} \\ \medskip
	    --- Margaret Mead \citep{Buxton:2007}
	\end{flushright}
\end{quote}
%\pagestyle{empty}

\hfill

\vfill

\begingroup
\let\cleardoublepage\relax
\chapter*{Epilog}
Wir versuchten mit dieser Arbeit ein Grundverständnis für Designmethoden und \ac{CSCW} Systeme zu bilden, um die Arbeitsweisen von Designer in Bezug auf digitale Medien kennenzulernen. Durch das nähere Betrachten von kollaborativen Designpraktiken und den Unterschieden, die beim Einsatz der wichtigsten Designmethode (die des Skizzierens) bei traditionellen und digitalen Arbeiten auftreten, konnten wir die Diskrepanzen erkennen, die beim Arbeiten mit digitalen Artefakten entstehen können. \\
Bestehende Systeme zeigten uns, dass zwar schon viele Versuche unternommen wurden, kollaboratives Arbeiten auf digitale Medien umzusetzen, man dabei jedoch immer wieder auf Barrieren stieß, da es bisher noch keinem gelang, eine effiziente und transparente Schnittstelle zwischen analogen und digitalen Objekten zu schaffen. \\
Unsere praktischen Erfahrungen mit \scribbler zeigten, dass noch einige Verbesserungen auf diesem Sektor notwendig sind, um diese Barrieren aufbrechen zu können. Jedoch bewegten wir uns mit unserem System einen Schritt in die richtige Richtung.
\endgroup