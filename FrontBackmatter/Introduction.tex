%*******************************************************
% Introduction
%*******************************************************
\cleardoublepage
%\pdfbookmark[-1]{Epilog}{Epilog}
\phantomsection
\addtocontents{toc}{\protect\vspace{\beforebibskip}} % to have the bib a bit from the rest in the toc
\addcontentsline{toc}{chapter}{\tocEntry{Einleitung}}
%\begin{quote}
%	\begin{flushright}{\slshape    
%	    zitat.} \\ \medskip
%	    --- Max Mustermann \citep{xx:xxxx}
%	\end{flushright}
%\end{quote}
%\pagestyle{empty}

\hfill

\vfill

\begingroup
\let\cleardoublepage\relax
\chapter*{Einleitung} \label{app:introduction}
In den letzten Jahrzehnten wurde viel Forschung betrieben, um die Unterstützung kollaborativer Arbeitssituationen durch technologische Mittel voran zu treiben. Vor Allem im Web stieg die Zahl der Tools zur entfernten Zusammenarbeit. Gleichzeitig musste man aber feststellen, dass die bequeme Onlinekollaboration auch Nachteile mit sich brachte. Simple, jedoch wichtige Kommunikationsformen, wie Augenkontakt, Gestik und Mimik fehlten. Aus diesem Grund können kollaborative Meetings bis dato immer noch nicht vollständig ersetzt werden. Besonders Designer verbringen viel Zeit in Meetings, um zusammen mit anderen Ideen zu erarbeiten. Ihr wichtigstes Werkzeug sind dabei Stift und Papier, mit deren Hilfe sie Notizen und einfache Skizzen anfertigen.\\
Da aber auch hier die Technisierung nicht halt machte, und Designer vermehrt mit digitalen Medien zu tun haben, kommt es oft zu ungewolltem Mehraufwand. Designs die nur elektronisch existieren, können zwar meist in einer geeigneten Form (z.B. mittels Projektor) präsentiert, jedoch Anmerkungen, Notizen oder neue Ideen nie direkt auf das digitale Design geschrieben oder gezeichnet werden. \\
Es existieren zwar bereits Systeme, die das kollaborative Skizzieren auf einem gemeinsamen Display ermöglichen, jedoch werden Skizzen an das jeweilige System gebunden, wodurch es unmöglich ist, auf beliebigen Inhalten zu arbeiten.

\medskip Die vorliegende Arbeit erläutert die Problemfelder dieser Situation und zeigt anhand des Projekts \scribbler einen möglichen Lösungsansatz. In den folgenden Kapiteln werden zunächst in \autoref{ch:research} bestehende Systeme beschrieben und im Zuge dessen, wichtige Anforderungen an ein solches System gesammelt. \autoref{ch:designTheorie} und \autoref{ch:kollaborativesDesign} erklären Arbeitsweisen von Designteams mittels Methoden und Verhaltensweisen in kollaborativen Settings. In \autoref{ch:DesignVSComputer} wird der Unterschied zwischen dem Arbeiten mit traditionellen und digitalen Medien spezifiziert und das anschließende Kapitel \autoref{ch:CSCWDesign}, beschäftigt sich mit der Definition von \ac{CSCW} und Groupware und betrachtet Eigenheiten und Implikationen. Die so gewonnenen Erkenntnisse werden dann im zweiten Teil der Arbeit, \autoref{ch:scribbler} veranschaulicht und ein Bezug zur Praxis hergestellt. 

\bigskip \emph{Anmerkung: Über \graffito{\(\clubsuit\)}die gesamte Arbeit werden Anmerkungen, seitlich durch das \begin{footnotesize}\(\clubsuit\)\end{footnotesize}-Symbol gekennzeichnet. Kommentare \graffito{Kommentar}werden ebenfalls seitlich platziert.}
\endgroup



