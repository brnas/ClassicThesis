%*******************************************************
% Introduction
%*******************************************************
\cleardoublepage
%\pdfbookmark[-1]{Epilog}{Epilog}
\phantomsection
\addtocontents{toc}{\protect\vspace{\beforebibskip}} % to have the bib a bit from the rest in the toc
\addcontentsline{toc}{chapter}{\tocEntry{Einleitung}}
%\begin{quote}
%	\begin{flushright}{\slshape    
%	    zitat.} \\ \medskip
%	    --- Max Mustermann \citep{xx:xxxx}
%	\end{flushright}
%\end{quote}
%\pagestyle{empty}

\hfill

\vfill

\begingroup
\let\cleardoublepage\relax
\chapter*{Einleitung} \label{app:introduction}
In den letzten Jahrzehnten wurde viel Forschung betrieben, um die Unterstützung kollaborativer Arbeitssituationen durch technologische Mittel voran zu treiben. Vor Allem im Web stieg die Anzahl an Tools zur entfernten Zusammenarbeit an. Gleichzeitig musste man aber feststellen, dass die bequeme Onlinekollaboration auch Nachteile mit sich brachte. Einfache Eigenschaften die bei traditionellen Gruppenarbeiten existieren, wie Augenkontakt, Gestik etc. verschwanden. Aus diesem Grund haben kollaborative Meetings bis dato nicht an Beliebtheit verloren. Besonders Designer verbringen viel Zeit in Meetings um zusammen mit anderen Ideen zu erarbeiten. Ihr wichtigstes Werkzeug dafür sind Stift und Papier, mit dessen Hilfe sie Notizen und einfache Skizzen anfertigen können.\\
Da aber auch hier die Technisierung nicht halt machte und Designer vermehrt mit digitalen Medien zu tun haben, kommt es oft zu ungewolltem Mehraufwand. Designs die nur elektronisch existieren, können zwar meist in einer geeigneten Form (z.B. mittels Projektor) präsentiert werden, jedoch können Anmerkungen, Notizen oder neue Ideen nie direkt auf das digitale Design geschrieben oder gezeichnet werden. \\
Es existieren zwar bereits Systeme, die das kollaborative Skizzieren auf einem gemeinsamen Display ermöglichen, jedoch werden Skizzen an das jeweilige System gebunden, was es unmöglich macht jegliche Inhalte zu bearbeiten.

\medskip Mit der vorliegenden Arbeit, erläutern wir die Problemfelder dieser Situation und zeigen mit dem Projekt \scribbler einen möglichen Lösungsansatz. In den folgenden Kapiteln wollen wir zunächst in \autoref{ch:research} bestehende Systeme beschreiben und im Zuge dessen, wichtige Anforderungen an ein solches System sammeln. \autoref{ch:designTheorie} und \autoref{ch:kollaborativesDesign} erklären Arbeitsweisen von Designteams mittels Methoden und Verhaltensweisen in kollaborativen Settings. In \autoref{ch:DesignVSComputer} spezifizieren wir den Unterschied zwischen dem Arbeiten mit traditionellen und digitalen Medien. Anschließend charakterisieren wir in \autoref{ch:CSCWDesign} gruppenbasierte Systeme, bevor wir in \autoref{ch:scribbler} alle gewonnen Erkenntnisse mit Hilfe unseres Prototypen veranschaulichen wollen.

\bigskip \emph{Anmerkung: Über \graffito{\(\clubsuit\)}die gesamte Arbeit werden Anmerkungen, seitlich durch das \begin{footnotesize}\(\clubsuit\)\end{footnotesize}-Symbol gekennzeichnet. Kommentare \graffito{Kommentar}werden ebenfalls seitlich platziert.}
\endgroup



